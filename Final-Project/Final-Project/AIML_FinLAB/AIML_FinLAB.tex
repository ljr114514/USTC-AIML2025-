\documentclass[12pt,a4paper]{ctexart}
\ctexset{
  section = {
    format = \raggedright\Large\bfseries
  }
}
\usepackage{amsmath}
\usepackage{graphicx}
\usepackage{amsfonts}
\usepackage{amsthm}
\usepackage{subcaption}
\usepackage{float}
\usepackage{booktabs}
\usepackage{xcolor}
\usepackage[utf8]{inputenc}
\usepackage{amsmath}
\usepackage{amssymb}


\title{USTC AI3002 Final-Project 实验报告}
\author{王梓浩 \and 刘谨睿 \and 崇时瑞}

\begin{document}
\maketitle
\newpage
\section{环境配置}

%我的环境配置
崇时瑞: \begin{itemize}
    \item Pytorch: 2.6.0+cu126
    \item CUDA 可用
    \item 显卡: NVIDIA GeForce RTX 4070 Laptop GPU
\end{itemize}

\begin{figure}[h]
    \centering    
    \includegraphics[width=0.8\textwidth]{1}
    \caption{相关配置截图}
\end{figure}

%需要填写的环境配置

%需要填写的环境配置
\newpage
\section{Warm-up}
\subsection{Monte Carlo Learning}
\subsubsection{First-visit MC核心思想}
原理简述:\\
First-Visit MC 通过计算状态 $s$ 在多个回合中首次出现后的平均回报来估算其价值。
其核心逻辑基于大数定律,即经验平均值在样本足够多时会收敛于期望值。\par
算法核心步骤:

\begin{enumerate}
    \item {采样}:按照策略 $\pi$ 生成一个完整的回合(Episode)。
    \item {识别}:在该回合中,仅标记每个状态 $s$ 第一次出现的时刻 $t$。
    \item {更新}:计算从 $t$ 开始的回报 $G_t = \sum_{k=0}^{T-t-1} \gamma^k R_{t+k+1}$,并更新价值函数:
    \begin{equation}
        V(S_t) \leftarrow V(S_t) + \alpha [G_t - V(S_t)]
    \end{equation}
\end{enumerate}

特点总结:
\begin{itemize}
    \item 无偏性:直接利用实际回报,不依赖其他状态的估计值。
    \item 收敛慢:必须完成整个回合才能更新,方差通常较大。
\end{itemize}

\subsubsection{Black Jack 实验结果及分析}
本实验对比了不同训练轮数下,First-Visit MC 算法对 Blackjack 环境价值函数的估计效果。图 \ref{fig:mc_results} 展示了在是否有 Usable Ace 的情况下,10,000 次与 500,000 次迭代的结果。

\begin{figure}[htbp]
    \centering

    \begin{subfigure}[h]{0.48\textwidth}
        \centering
        \includegraphics[width=\textwidth]{MC_10k_ace.png}
        \caption{10,000 Episodes (Usable Ace)}
    \end{subfigure}
    \hfill
    \begin{subfigure}[h]{0.48\textwidth}
        \centering
        \includegraphics[width=\textwidth]{MC_10k_no_ace.png}
        \caption{10,000 Episodes (No Usable Ace)}
    \end{subfigure}
    
    \vspace{1em}
    
    \begin{subfigure}[h]{0.48\textwidth}
        \centering
        \includegraphics[width=\textwidth]{MC_500k_ace.png}
        \caption{500,000 Episodes (Usable Ace)}
    \end{subfigure}
    \hfill
    \begin{subfigure}[h]{0.48\textwidth}
        \centering
        \includegraphics[width=\textwidth]{MC_500k_no_ace.png}
        \caption{500,000 Episodes (No Usable Ace)}
    \end{subfigure}
    
    \caption{不同迭代次数下 MC 算法估算的价值函数曲面对比}
    \label{fig:mc_results}
\end{figure}

\par 结果讨论
\begin{itemize}
    \item 样本量对平滑度的影响:从图中可以观察到,10k 次迭代时曲面存在明显的锯齿状波动,这是由于 MC 方法方差较大导致的。当迭代次数增加至 500k 时,价值函数曲面变得平滑且稳定,证明了算法的收敛性。
    \item 状态价值分布:在所有情况下,玩家总和达到 20 或 21 时,状态价值均接近 1.0(红色区域),而当庄家明牌为 A 且玩家总和较低时,价值明显降低。
    \item Usable Ace 的作用:拥有 Usable Ace 的曲面(a, c)在中间区域比无 Ace(b, d)更加平缓,反映了 A 牌作为 11 点使用时提供的额外生存机会。
\end{itemize}

\subsection{Temporal-Difference Learning}

\subsubsection{两种算法在 CliffWalking 的训练曲线对比}

在 CliffWalking 环境中,我们对比了 Q-learning(Off-policy)与 SARSA(On-policy)的表现。图 \ref{fig:td_comparison} 展示了两者的每回合回报(Episode Reward)与回合步数(Episode Length)的收敛过程。

\begin{figure}[htbp]
    \centering
    % 第一行:Reward 对比
    \begin{subfigure}[b]{0.48\textwidth}
        \centering
        \includegraphics[width=\textwidth]{episode_stats_q_learning_reward.png}
        \caption{Q-learning 每回合回报}
    \end{subfigure}
    \hfill
    \begin{subfigure}[b]{0.48\textwidth}
        \centering
        \includegraphics[width=\textwidth]{episode_stats_sarsa_reward.png}
        \caption{SARSA 每回合回报}
    \end{subfigure}

    \vspace{1em}

    % 第二行:Length 对比
    \begin{subfigure}[b]{0.48\textwidth}
        \centering
        \includegraphics[width=\textwidth]{episode_stats_q_learning_length.png}
        \caption{Q-learning 回合步数}
    \end{subfigure}
    \hfill
    \begin{subfigure}[b]{0.48\textwidth}
        \centering
        \includegraphics[width=\textwidth]{episode_stats_sarsa_length.png}
        \caption{SARSA 回合步数}
    \end{subfigure}
    
    \caption{Q-learning 与 SARSA 在 CliffWalking 环境下的训练统计对比}
    \label{fig:td_comparison}
\end{figure}

实验观察:
\begin{itemize}
    \item 收敛速度:从步数曲线可以看出,两种算法在前 100 个回合内均能迅速将步数从数百步降低至 50 步以内,体现了 TD 学习的高效性。
    \item 回报波动:对比图 (a) 与 (b),Q-learning 在收敛后的回报曲线上存在密集的负向“深坑”(Spikes),即即便在后期也经常触发 $-100$ 的掉落惩罚。相比之下,SARSA 的收敛曲线更加平稳,其平滑后的平均回报明显高于 Q-learning。
\end{itemize}

\subsubsection{SARSA 与 Q-learning 的差异分析}

结合“更安全路径 vs 更激进路径”的现象,两者的核心差异在于对 探索风险的处理方式:

\begin{enumerate}
    \item Q-learning (Off-policy) 的激进策略:
    Q-learning 学习的是目标策略(Optimal Policy),即沿着悬崖边走的最近路径。然而,由于训练中存在 $\epsilon$-greedy 探索,智能体有概率随机动作。在悬崖边进行随机动作极易导致掉落。因此,虽然它找到了理论最优路径,但在训练过程中的在线表现较差。

    \item SARSA (On-policy) 的安全策略:
    SARSA 在更新价值函数时考虑了当前的探索行为。它“意识到”在悬崖边探索会导致惨重损失,因此它学习到了一条虽然路径稍长、但远离悬崖边缘的安全路径。这使得 SARSA 在存在探索随机性的情况下,能够获得更高的累积奖励。
\end{enumerate}

结论:Q-learning 追求的是“理论最优”,适用于环境模型固定且执行时无随机扰动的场景;而 SARSA 追求的是“行为最优”,在探索阶段或环境具有随机性时表现得更加稳健。

\subsection{推荐选做部分}

\newpage
\section{Main Task}


\subsection{利用MCTS进行的模型优化}
\subsubsection{核心逻辑实现}
在 AlphaZero 框架中,我们利用MCTS承担了策略改进器的角色。我们通过在每个状态下进行多次模拟搜索,得到一个比纯神经网络输出更稳健的动作概率分布。为了解决 $12 \times 12$ 大规模棋盘带来的数值不稳定问题,我们在代码中实现了带有温度参数的 Log-Sum-Exp 稳定逻辑。


\begin{quote}
\small
\begin{verbatim}
def get_action_probs(self, state, temp=1.0):
    # 执行 n_simulations 次模拟搜索 (n=100)
    for _ in range(self.n_simulations):
        self._simulate(state)
        
    counts = np.array([child.visit_count for child in root.children.values()])
    actions = list(root.children.keys())

    # 针对温度极低的情况进行分支处理,防止除以零
    if temp < 1e-3:
        probs = np.zeros(self.board_size**2)
        probs[actions[np.argmax(counts)]] = 1.0
        return probs
    
    # 核心优化:使用 Log 空间计算防止数值溢出
    log_counts = np.log(counts + 1e-8) * (1.0 / temp)
    exp_counts = np.exp(log_counts - np.max(log_counts))
    probs_on_actions = exp_counts / np.sum(exp_counts)
    
    full_probs = np.zeros(self.board_size**2)
    full_probs[actions] = probs_on_actions
    return full_probs
\end{verbatim}
\end{quote}

\subsubsection{实验结果与效率分析}

\textbf{1. 训练吞吐量统计} \par
在本实验环境下,针对 $12 \times 12$ 规模的五子棋任务,系统在 12 小时的连续训练中完成了约 3800 个自博弈回合(Episodes)。虽然单回合生成的耗时较长(平均约 11.4 秒),但这保证了每一个样本都经过了深度的 MCTS 搜索(每次动作模拟 100 次),从而提供了高质量的梯度方向。

\textbf{2. 模型性能评估} \par
尽管训练步数有限,但模型的收敛质量超出了预期。经过约 3,000 轮的神经网络参数优化后,我们将该模型与执行完全随机策略的智能体进行了 100 场对局测试。
结果显示:模型达到了 100\% 的胜率。

这表明模型已经从最初的随机噪声中脱颖而出,成功掌握了五子棋的基础落子逻辑与防守技巧,验证了 MCTS 引导下的策略提升逻辑在 $12 \times 12$ 棋盘上的有效性。由与时间有限,以及MCTS方法的开销较大,后续的训练并未继续。
\end{document}